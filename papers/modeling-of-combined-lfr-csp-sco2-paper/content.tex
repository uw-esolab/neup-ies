
%\section{How to Use this Template}

%The template details the sections that can be used in a manuscript. Note that the order and names of article sections may differ from the requirements of the journal (e.g., the positioning of the Materials and Methods section). Please check the instructions on the authors' page of the journal to verify the correct order and names. For any questions, please contact the editorial office of the journal or support@mdpi.com. For LaTeX-related questions please contact latex@mdpi.com.
%The order of the section titles is: Introduction, Materials and Methods, Results, Discussion, Conclusions for these journals: aerospace,algorithms,antibodies,antioxidants,atmosphere,axioms,biomedicines,carbon,crystals,designs,diagnostics,environments,fermentation,fluids,forests,fractalfract,informatics,information,inventions,jfmk,jrfm,lubricants,neonatalscreening,neuroglia,particles,pharmaceutics,polymers,processes,technologies,viruses,vision

%%%%%%%%%%%%%%%%%%%%%%%%%%%%%%%%%%%%%%%%%%%%%%%%%%%%%%%%%%%%%%%%%%%%%%%%%%%%%%%%%%%%%%%
\section{Introduction}

%The introduction should briefly place the study in a broad context and highlight why it is important. It should define the purpose of the work and its significance. The current state of the research field should be reviewed carefully and key publications cited. Please highlight controversial and diverging hypotheses when necessary. Finally, briefly mention the main aim of the work and highlight the principal conclusions. As far as possible, please keep the introduction comprehensible to scientists outside your particular field of research.  %Please use the command \citep{} for the following MDPI journals, which use author--date citation: Administrative Sciences, Arts, Econometrics, Economies, Genealogy, Histories, Humanities, IJFS, Journal of Intelligence, Journalism and Media, JRFM, Languages, Laws, Religions, Risks, Social Sciences.


Recompression supercritical CO$_2$ Brayton cycles are promising cycle configurations offering higher efficiencies, compact design, and reduced turbomachinery cost while operating with non toxic working fluid. High efficiencies, in some simple recompression cycles reaching 50\% efficiency \cite{turchi_2013,wright_2009}, allows these cycles to be a competitive alternative to steam Rankine and air Brayton cycles over a range of temperatures.
\bl{Updated above as sCO2 can also be used for Coal - it is a cycle vs cycle argument rather than a heat source vs heat source argument -BW}
Various sCO$_2$ Brayton cycles have been modeled with the recompression cycle having the most efficiency advantages over other proposed cycle arrangements \cite{turchi_2013,ahn_2014,wang_2018}.  Due to the benefits of sCO$_2$ Brayton cycles, the United States Department of Energy is investigating these conversion cycles for use with heat sources including nuclear and solar \cite{doe_2012}. Multiple project funding opportunities are established with the National Energy Technology Laboratory offering 144 million dollar reward for demonstration and performance verification of a sCO$_2$ Brayton cycle \cite{netl_2016} and the Office of Energy Efficiency \& Renewable Energy offering 2.6 million dollar reward in their Brayton Energy project for a integrated CSP receiver, TES, and power block in one sCO$_2$ Brayton system \cite{seto_2015}. Utilizing complementary technologies, specifically solar concentrating power and lead-cooled fast reactors, can offset the drawbacks of each. Coupling these technologies into an interconnected cycle allows for consistent generation, independent of weather or time of day, and thermal storage for high dispatchability during high grid demand periods.

\bl{not a big issue but do we want to describe as sCO2 or sCO2 Brayton? In project nomenclature to date I have used sCO2 with the logic that sCO2 'isnt really a Brayton cycle' but both are common -BW}
A CSP has an array of mirrors concentrating solar rays towards a receiver to generate thermal energy therefore causing a dependency on weather conditions and time of day. Previous research has modeled CSP technology as the thermal source for Brayton sCO$_2$ cycles with promising results in efficiency gains and high temperature thermal energy storage \cite{turchi_2013, iverson_2013, ho_2015, wang_2018, neises_2020}. Thermal energy gathered from the CSP is stored in thermal energy storage which is dispatched into the sCO$_2$ Brayton cycle when grid demand increases. With a comparable temperature to the hot TES, 560 $^{\circ}$C, a LFR is capable of transferring heat to thermal storage for later dispatch. An LFR uses fission reactions to heat a heat transfer fluid, in this case liquid lead, to temperatures of 650 $^{\circ}$C. Nuclear sCO$_2$ Brayton cycles have been studied with similar gains in efficiency seen in CSP sCO$_2$ Brayton cycles \cite{ dostal_2004, luo_2020}. Physical testing of simple sCO$_2$ Brayton cycles is researched at Sandia National Labs in Arvada, Colorado, with break-even conditions, where the turbine produces enough power to offset compressor demand, being achieved \cite{wright_2011}. In addition to break-even conditions, research from the Supercritical CO$_2$ Brayton Cycle Integral Experiment Loop, in Korea Atomic Energy Institute, achieved electrical generation in a simple sCO$_2$ Brayton cycle \cite{cha_2016}. 

\bl{I would bullet out the next three citations-BW}
Various studies on complimentary solar-nuclear systems have been accomplished with several reviewed as follows: 

\begin{itemize}
    \item	Monnerie et al. (2011): In search of an alternative to the typical fossil fuel-based process, reports on synthesizing hydrogen with complimentary solar-nuclear technologies being utilized as consistent heat sources for chemical decomposition of sulphuric acid to aid in simple, low-temperature electrolysis \cite{monnerie_2011}. 
    \item   Curtis J.D. (2015): Massachusetts Institute of Technology thesis reports cycle configuration, performance, and development of complimentary solar-nuclear systems with a focus on shale oil extraction and production from kerogen deposits \cite{curtis_2015}. 
    \item   Wang et al. (2020): Implements a combined solar and nuclear plant discussing a sCO$_2$ Brayton recompression cycle layout with an emphasis on a cycle design's performance to varying solar irradiance and demonstration of feasibility \cite{wang_2020}. 
\end{itemize}

The economic and physical feasibility of producing shale oil and hydrogen with complimentary solar-nuclear cycles, as discussed in the Curtis J.D. (2015) thesis and Monnerie et al. (2011) article, is not within the scope of sCO$_2$ Brayton recompression cycle electrical generation and storage presented in this paper.

The single cycle configuration in Wang et al. (2020) has a higher-temperature operating salt, 67\%KCl-33\%MgCl$_2$, as the heat transfer fluid in the CSP which is capable of temperatures in the range of 450 - 1400 $^{\circ}$C. The lower operating point of the studied solar salt in this paper, 60\%NaNO$_3$-40\%KNO$_3$ with a temperature range of 250 - 585 $^{\circ}$C, is employed because the technology is more established when compared to KCl-MgCl$_2$ \cite{turchi_2018}. The higher temperature salt requires preheating which is achieved prior to the CSP heat addition by a small modular lead-cooled fast reactor. The small modular LFR operates at a low temperature and a fixed location when compared to the higher temperature lead and variable location that the Westinghouse LFR studied in this paper is capable of. Due to the similar operating temperatures of the CSP and LFR, multiple cycle configurations are studied that are not possible with the similar components in Wang et al. (2020). Specifically studied in this paper are the effects on cycle and heat storage efficiency of different cycle configurations while having the LFR serve a dual purpose of electrical generation and a supplementary CSP TES charging technique. This paper will provide an overview of contending recompression sCO$_2$ Brayton cycles with varied positioning of complimentary CSP and LFR heat additions in the cycle. Additionally the location of where heat is drawn from the sCO$_2$ Brayton cycles to be stored in thermal energy storage is studied with the results discussed.


%Monnerie et al. (2011), in search of an alternative to the typical fossil fuel-based process, reports on combining nuclear and solar energy in the same plant to produce hydrogen \cite{monnerie_2011}.  Extraction and production of fuels using complimentary solar-nuclear systems is additionally studied in the Curtis 2015 thesis from Massachusetts Institute of Technology where cycle configuration, performance, and development of shale oil extraction from kerogen deposits is reported \cite{curtis_2015}. Utilizing complimentary solar-nuclear cycles for production of hydrogen or shale oil with economical feasibility of the system discussed, in contrast to a fucus on electrical generation and storage in a sCO$_2$ Brayton recompression cycle, is not within the scope of the presented research in this paper.

%Implementation of a combined solar and nuclear plant, modeled in the Wang et al. (2020) shows a single sCO$_2$ Brayton recompression cycle layout with an emphasis on a cycle design's performance to varying solar irradiance and demonstration of feasibility \cite{wang_2020}.  This single cycle configuration has a dissimilar higher-temperature operating salt, KCl-MgCl$_2$, as the heat transfer fluid in the CSP with the small modular lead-cooled fast reactor's main purpose as a preheater to the CSP heat addition instead of being utilized as an TES charging alternative. Presented in this paper are the effects on cycle and heat storage efficiency of alternative heat addition locations while having the LFR serve a dual purpose of electrical generation and a supplementary CSP TES charging technique. 
%This paper will provide an overview of contending recompression sCO$_2$ Brayton cycles with varied positioning of complimentary CSP and LFR heat additions in the cycle. Additionally the location of where heat is drawn from the sCO$_2$ Brayton cycles to be stored in thermal energy storage is studied and the results discussed.
\bl{I think the last citation merits its own paragraph and a little further discussion (1-2 more sentences). You could start by saying that it is prior work on the LFR-CSP-sCO2 concept. Then describe its cycle configuration with the high temperature salt. Then in the following sentence (the stuff you currnetly have highlighted) you can say that your work is investigating higher lead temperatures than Wang (because of Westinghouse LFR) and lower temperature salt (technology readiness) and this leads to very different cycle configurations and options than Wang, as will be discussed. The overall tone is that you work recognizes and builds on Wang, rather than needing to justify its existance despite Wang - BW}


 
%%%%%%%%%%%%%%%%%%%%%%%%%%%%%%%%%%%%%%%%%%%%%%%%%%%%%%%%%%%%%%%%%%%%%%%%%%%%%%%%%%%%%%%
\section{Materials and Methods}

%Materials and Methods should be described with sufficient details to allow others to replicate and build on published results. Please note that publication of your manuscript implicates that you must make all materials, data, computer code, and protocols associated with the publication available to readers. Please disclose at the submission stage any restrictions on the availability of materials or information. New methods and protocols should be described in detail while well-established methods can be briefly described and appropriately cited.

%Research manuscripts reporting large datasets that are deposited in a publicly avail-able database should specify where the data have been deposited and provide the relevant accession numbers. If the accession numbers have not yet been obtained at the time of submission, please state that they will be provided during review. They must be provided prior to publication.

%Interventionary studies involving animals or humans, and other studies require ethical approval must list the authority that provided approval and the corresponding ethical approval code.
%\begin{quote}
%This is an example of a quote.
%\end{quote}

% **************************************

%MW: I've fixed spelling in a few places. You can install a spell checker to highlight misspelled words. I use the "Code spell checker" extension.


\subsection{Cycle Component Modeling}
%======================================================================================
Components present in the cycles are modeled using various techniques and are discussed in more detail below. Turbines and compressors are analyzed using isentropic efficiencies. Counter-flow heat exchangers are modelled using the effectiveness-NTU method while simplified ``black box'' heat exchangers that use a simple energy balance for state point calculations are used in lieu of more detailed component models where data is available. The lead-cooled fast reactor is assumed to be a black box heat exchanger because of the constant heat input and state points on the sCO$_2$ inlet and outlet are provided. The concentrating solar power cycle is modeled with necessary components including hot and cold TES, receiver, pumps and counter-flow heat exchangers.



\subsubsection{Turbines and Compressors }
%--------------------------------------------------------------------------------------

Turbines and compressors are modeled for each cycle using constant isentropic efficiency values which are summarized in Table \ref{tab-cycle-constants}. Turbines take the high pressure sCO$_2$ and expand it through a series of blades allowing a production of energy, while compressors input mechanical energy to increase the pressure of the sCO$_2$. The turbines and compressors are assumed to be at steady state, exchange no heat with the surroundings, and have single inlet and outlet streams. Using this estimate, along with a known low and high side pressures, temperature and enthalpy outlets of the turbine and compressor are calculated \cite{klein_nellis_2011}. 

%Power calculations are initially done on ideal, reversible compressors or turbines then are scaled by the isentropic efficiencies to provide an estimate for a realizable component.\mw{this is kind of an awkward way to describe what's happening. I think most readers are familiar with isentropic efficiencies and this can be omitted.}

\subsubsection{Black Box and Counter-Flow Heat Exchangers}
%--------------------------------------------------------------------------------------

Black box heat exchangers are simplified heat exchangers which have no approach temperature or pinch point and are modeled as a perfect heat transfer into or out of the cycle. These heat exchangers use an energy balance with mass flow inlet energy, heat input or output, and mass flow outlet energy. The energy balance equation used for all black box heat exchangers is Equation \ref{eq-black}.

%\mw{This is for a single flow stream. Why would there be $\dot{Q}_{in}$ and $\dot{Q}_{out}$ in a single flow stream? Also, the variable should be consistent with the figure, which would be $\dot{Q}_{HX}$. -BW}


\begin{equation}
    \label{eq-black}
    \dot{m} \cdot h_{in} + \dot{Q}_{HX} = \dot{m} \cdot h_{out} ,
\end{equation}

In this equation the energy input to the system is on the left hand side with $\dot{m}$ multiplied by $h_{in}$ being energy from the mass flow while $\dot{Q}_{HX}$ is heat transfer directly into, positive, or out of, negative, the flow from an outside source. The right hand side of the equation is heat leaving the black box heat exchanger with $\dot{m}$ and enthalpy of $h_{out}$. Black box energy balances are used in three situations, the receiver, LFR heat exchanger, and PC heat exchanger. These heat exchangers are not exhaustively modeled because the state points on the inlet and outlet are defined by design parameters.

Counter-flow heat exchangers are modeled with two fluids flowing in opposite directions exchanging heat from the hot side to the cold side. The temperatures of the hot and cold flows on either side of the the heat exchanger have a temperature difference known as an approach temperature. A diagram showing a simplified counter-flow heat exchanger is illustrated in Figure \ref{counter-flow-hx}.

\end{paracol}
\begin{figure}[H] 
    \widefigure
    \includegraphics[width=10 cm]{Definitions/counter-flow-hx.pdf}
    \caption{Simple counter-flow heat exchanger diagram. \label{counter-flow-hx}}
\end{figure}
\begin{paracol}{2}
\linenumbers
\switchcolumn

Additional assumptions of the counter-flow heat exchanger model are: no heat loss to the surroundings, no pressure drops across the heat exchangers, and no fouling resistances. In Figure \ref{counter-flow-hx} the subscript 'out' denotes where the streams are leaving, 'in' denotes the entering streams, 'c' and 'h' signify cold and hot streams respectively, $\dot{Q}_{HX}$ is the total heat transfer from the hot to cold stream, and $\dot{Q}_{surr}$ is the heat transfer to the surroundings.

Counter-flow heat exchanger calculations require two known state points, fluid libraries, mass flow rate of hot and cold side, and a specified approach temperature. In the modeled cases, the approach temperature is set to value of $10^{\circ}$C, based off prior model development of sCO$_{2}$ Brayton cycle heat exchangers \cite{seidel_2010_model_development}. The fluid libraries referenced are built into EES for Carbon Dioxide and Salt (60\% NaNO$_3$ 40\% KNO$_3$) \cite{pacheco_1995_salt_properties,roland_1996_co2_properties}. 

To analyze the counter-flow heat exchanger a side is chosen, usually the high side, to start the calculations. The approach temperature is initially subtracted from the hot stream on the high temperature side to find the missing cold temperature according to Equation \ref{temp_h}. 

\begin{equation}
   \label{temp_h}
    T_{out,c} = T_{in,h}-\Delta_{T},
\end{equation}

Where T$_{out,c}$ is the cold stream outlet temperature and T$_{in,h}$ is the hot stream inlet temperature. Knowing the two state points allows for the enthalpy out to be found using correlations from the fluid property libraries. This enthalpy then allows for the heat transfer of the heat exchanger to be found with Equation \ref{heattrans_h}.

\begin{equation}
    \label{heattrans_h}
     \dot{Q}_{HX} = \dot{m}_{c}(h_{out,c}-h_{in,c}),
 \end{equation}

 Where $\dot{Q}_{HX}$ is the total heat transfer rate from the hot stream to the cold stream, $\dot{m}_{c}$ is the mass flow rate of the cold stream, $h_{out,c}$ is the enthalpy at the outlet of the cold side, and $h_{in,c}$ is the inlet of the cold side.
 The known heat transfer of the counter-flow heat exchanger can then solve for the enthalpy out of the hot stream, h$_{out,h}$. This is accomplished with Equation \ref{enthalpy_h}.

 \begin{equation}
    \label{enthalpy_h}
     h_{out,h} = h_{in,h} - \frac{\dot{Q}_{HX}}{\dot{m}_{h}},
 \end{equation}

Knowing the hot stream enthalpy out allows for all states to be set on the outlets and inlets of the counter-flow heat exchanger. The temperature difference of the low side is then checked to ensure that it is larger than the approach temperature, defined at $10^\circ$C. If the temperature difference on the low side is smaller than the approach temperature, the same computations are carried with the low side as the starting point.

Knowing the state points on all inlets and outlets of the counter-flow heat exchanger allows for the heat exchanger performance metrics to be calculated. Performance metrics include effectiveness, capacitance ratio, UA, and NTU for heat exchangers. Effectiveness is the ratio of the actual heat transfer rate to the maximum heat transfer rate, or a perfect heat exchanger with no approach temperature. Assuming the approach temperature is on the high side, the maximum heat transfer rate, $\dot{Q}_{max}$ is found with the maximum enthalpy. Maximum enthalpy of the cold stream is found with correlations by setting the temperature to T$_{in,h}$ with same pressure on the cold outlet. Using the maximum enthalpy, $h_{max}$, the maximum heat transfer rate is calculated using Equation \ref{heattrans_max}.

\begin{equation}
    \label{heattrans_max}
    \dot{Q}_{max} = \dot{m}_{c}(h_{max}-h_{in,c}),
\end{equation}

Calculating the maximum heat transfer rate allows for effectiveness to be calculated using the ratio in Equation \ref{effective}.

\begin{equation}
    \label{effective}
    \varepsilon = \frac{\dot{Q}_{HX}}{\dot{Q}_{max}},
\end{equation}

All of the prior equations are carried out in a built-in function within EES. EES is an iterative solver, therefore as long as there is a feasible solution, the functions can take any of the four state points around the heat exchanger and converge on a solution.

After the effectiveness is solved for, capacitance ratio is necessary. The capacitance ratio is defined as the average minimum capacitance rate, $\dot{C}_{min}$, over the average maximum capacitance rate, $\dot{C}_{max}$. Average capacitance rates for the hot and cold streams are found by multiplying the addition of the specific heat at the inlet and outlet of the stream by the mass flow and dividing by two as seen in Equation \ref{avg_cap}.

\begin{equation}
    \label{avg_cap}
    \dot{C}_{avg} = \frac{\dot{m}(c_{in}+c_{out})}{2},
\end{equation}

Where $C_{avg}$ is the average capacitance rate across the hot or cold stream and $c_{in}$ and $c_{out}$ is the specific heat at the inlet and outlet respectively. Specific heat is found using library correlations. Once both average capacitances are calculated for the hot and cold streams, the one will have a larger value, $\dot{C}_{max}$, and one will have a smaller value,  $\dot{C}_{min}$. These maximum and minimum values are then used to find the capacitance ratio, $CR$, in Equation \ref{cap_ratio}.

\begin{equation}
    \label{cap_ratio}
    CR = \frac{\dot{C}_{min}}{\dot{C}_{max}},
\end{equation}

Using values of effectiveness and capacitance ratio, the effectiveness-NTU method can be employed to find the number of transfer units, $NTU$, and the conductance, $UA$ \cite{klein_nellis_2011,nellis_klein_2008}. 



\subsubsection{Lead-Cooled Fast Reactor}
%--------------------------------------------------------------------------------------
Lead-cooled fast reactors use energy from a controlled nuclear reaction to heat molten lead. This lead is used to cool the core and transfer heat into the sCO$_2$ Brayton power cycle \cite{smith_2016_lfr_background,alemberti_2013_lfr_overview}. The lead-cooled fast reactor is assumed to be a black box heat transfer and is labeled in the cycle models LFR HX. The inlet, outlet and heat transfer rates are provided by our industry partner, Westinghouse, making the black box simplification viable. The energy balance for the black box assumption can be seen in Equation \ref{eq-lfr-black-box}.

\begin{equation}
    \label{eq-lfr-black-box}
    \dot{m} \cdot h_{inlet} + \dot{Q}_{LFRHX} = \dot{m} \cdot h_{outlet},
\end{equation}

%\mw{If mass is conserved, use a single $\dot{m}$ - BW}

Where the left hand side, $\dot{m}$, $h_{inlet}$, and $\dot{Q}_{LFRHX}$, is the energy into the flow and the right hand side, $\dot{m}$ and $h_{outlet}$, is the energy brought out from the flow of sCO$_2$. The amount of energy transferred into the cycle, $\dot{Q}_{LFRHX}$, is set at $950$ MW, and outlet temperature of the sCO$_2$ from the LFR HX is set at a value of $595^{\circ}$C. 
%\mw{Make sure numbers are in math mode -BW}
The outlet temperature of the LFR is specified, accounting for the high temperature material limits on the LFR lead side. LFR sCO$_2$ inlet low temperature has a lower bound of $340^{\circ}$C and an optimal value of $400^{\circ}$C, as provided by Westinghouse.
%\mw{How do we know that 400C is the optimal temperature? This information is given to us by Westinghouse, right? -BW} 
Increasing the sCO$_2$ LFR inlet temperature above this optimal value leads to diminishing returns on the LFR efficiency.


\subsubsection{Concentrating Solar Power Cycle}
%--------------------------------------------------------------------------------------

The CSP salt cycle modeled in this paper is composed of hot and cold thermal energy storage (TES), pumps, receiver, sCO$_2$-to-salt counter-flow heat exchanger (C2S), and CSP counter-flow heat exchanger (CSP HX). The diagram for this CSP salt cycle is seen in Figure \ref{csp}. 

%\mw{Best practice is to allow figures to float. To do this, specify the location as [htb] ("here, top, or bottom"). I've done a search-replace for all figures to do this. -BW}

% \end{paracol}


\begin{figure}[H] 
    %[htb]
    \widefigure
    \includegraphics[width=10 cm]{Definitions/csp.pdf}
    \caption{Diagram for CSP cycle with cold and hot thermal energy storage, pumps, and csp black box heat input\label{csp}}
\end{figure}
% \begin{paracol}{2}
% \linenumbers
% \switchcolumn

The CSP salt cycle uses 60\% sodium nitrate, NaNO$_3$, and 40\% potassium nitrate, KNO$_3$, 'solar salt' as the heat transfer fluid. Solar salt stored in the hot TES can be dispatched on demand through the CSP HX when grid demand increases and held when grid demand is low. Current CSP salt cycles heat solar salt with receivers and store it in hot TES tanks at temperatures around 565$^{\circ}$C. Future CSP salt cycles are hypothesized to have bulk hot TES temperatures of up to 720$^{\circ}$C, but due to high temperature limitations with the LFR outlet, the hot TES temperature is set at 560$^{\circ}$C for all modeled cycles \cite{mehos2017concentrating}. The cold TES temperature takes on three different values according to cycle configuration capabilities: 390$^{\circ}$C, 410$^{\circ}$C, and 440$^{\circ}$C. In addition to the lower hot TES temperature, current CSP salt cycles lack a secondary option for charging the hot TES \cite{hamilton2020dispatch}. The studied CSP salt cycle has two TES charging options: a receiver, which generates heat from a heliostat field, and C2S heat exchanger, which draws excess heat from the sCO$_2$ Brayton cycle. While the hot TES is charging, the receiver and LFR are storing heat for later use when grid demand increases. The hot TES storage is not dispensing salt for use in the CSP cycle while charging.

The C2S heat exchanger is active in the 'charging' cycle operating modes, when the focus is on heat storage for later use. Pump 1 is actively moving solar salt from cold TES to hot TES through the C2S heat exchanger extracting heat from the sCO$_2$ Brayton cycle. Additionally, while the focus is on heat storage, and the heliostat field is inputting heat, pump 2 is actively transporting solar salt through the receiver to be stored in the hot TES.
%\mw{Is this necessarily true? Couldn't there be a condition where we'd like to dump energy from the LFR while simultaneously collecting from CSP? } 

'Non-charging' cycle operating modes are characterized by operations wherein the CSP salt cycle is discharging the hot TES, the C2S heat exchanger is not transferring heat, and the LFR is dispatching heat directly to generate electricity. When electrical generation is occurring, the heat input in the CSP salt cycle is modelled through a black box energy balance across states S6-A and S1-A with a heat addition of 750 MW from the heliostat field. The hot TES solar salt is moved through Pump 3 and transfers heat into the sCO$_{2}$ Brayton cycle through CSP HX to be converted into electricity. The cooled salt is stored in cold storage and moved through Pump 2 where the heat from the receiver is again transferred into the CSP cycle. 

When grid demand for electrical power increases, a series of operating modes are activated. During the highest demand times, cycle operation focuses on maximum electrical generation. This is achieved through the C2S being turned off for direct electrical production from the LFR and the hot TES is discharging heat through the CSP HX for electrical production in the sCO$_2$ Brayton cycle. As grid demand diminishes, CSP HX ramps down heat extraction until no power is being dispatched through the salt and the hot TES begins charging. During this process, the LFR gradually adds a larger fraction of heat input to the TES through C2S. This process continues until no electrical production is occurring in the cycle and all heat is stored in TES for later use.

\bl{I dont think the above covers the case where the salt discharges. It might be worth eliciting how the operating modes change as power demand increases. From highest to lowest demand. In all modes CSP will charge the salt is sun is available: (1) LFR and S2C both dispatch, maximum power (2) As power demand decreases, S2C ramps down until no power dispatched through salt (i.e. salt is charged) (3) LFR dispatches, no dispatch from S2C (4) LFR ramps down and gradually transfers greater and greater fraction of power to salt using C2S. (5) No electricity dispatched, all LFR heat passes through C2S -BW
}



\subsection{Standardization of Cycle Modeling}
%======================================================================================
%\mw{What do you mean by ``generalized?''-BW}

In order to draw a more direct comparison, the cycles are standardized in terms of isentropic efficiencies, heat exchanger approach temperatures, pressures, heat input, and pump constants. These values are summarized in Table \ref{tab-cycle-constants}.

\begin{specialtable}[H] 
    %[htbp]
    %\mw{Also changing the specialtable position to allow latex to better place tables.-BW}
    \caption{Standardized constant cycle parameters with definition, variable and set value. \label{tab-cycle-constants}}
    \begin{tabular}{L{0.5\linewidth}cc}
    \toprule
    \textbf{Parameter} & \textbf{Variable}	& \textbf{Design Point Value}\\
    \midrule
    \textit{Efficiencies}\\
    Main Compressor & $\eta_{MC}$		& 0.91 (-)\\
    Re-Compressor & $\eta_{RC}$		& 0.89 (-)\\
    Turbine & $\eta_{T}$		& 0.90 (-)\\
    Pumps 1-3 & $\eta_{P}$      & 0.90 (-)\\
    \midrule
    \textit{Approach Temperatures}\\
    Low Temperature Recuperator & $\delta_{LTR}$		& 10 ($^{\circ}$C)\\
    High Temperature Recuperator & $\delta_{HTR}$		& 10 ($^{\circ}$C)\\
    Concentrating Solar Power Heat Exchanger & $\delta_{CSPHX}$	& 10 ($^{\circ}$C)\\
    \midrule
    \textit{Pressures}\\
    Pressure Ratio & $PR$ & 3.27 (-)\\
    High Side Pressure & $P_{2A}$ & 28.8 (MPa)\\
    \midrule
    \textit{Heat Into System}\\
    Lead-Cooled Fast Reactor Heat Transfer & $\dot{Q}_{LFRHX}$ & 950 (MW)\\
    Concentrating Solar Power Heat Transfer & $\dot{Q}_{CSP}$ & 750 (MW)\\
    \midrule
    \textit{Temperature}\\
    Main Compressor Inlet & $T_{1A}$ & 40 ($^{\circ}$C)\\
    %Lead-Cooled Fast Reactor sCO$_{2}$ Low Temperature & $T_{4}$,$T_{1C}$,$T_{5A}$,$T_{4C}$ & 673.2 (K)\\
    Lead-Cooled Fast Reactor sCO$_{2}$ High Temperature & $T_{5}$,$T_{2C}$,$T_{6A}$,$T_{5C}$ & 595 ($^{\circ}$C)\\
    \midrule
    \textit{Pumps}\\
    Pressure Rise Across Pump & $\Delta_{P}$ & 3.726 (MPa)\\
    Pump Low Side Pressure & $P_{S5-B}$ & 3 (MPa)\\ 
    \bottomrule
    \end{tabular}
\end{specialtable}

The values displayed in Table \ref{tab-cycle-constants} are representative of LFR and CSP design while being consistent with design parameters given by our industry partner, Westinghouse Electric Company. 


In addition to standardized parameters, all cycles have identical recompression sides. The recompression side contains a precooler, low temperature recuperator, and two compressors; main compressor and recompressor. 
\mw{We need an overview here of our process for identifying which cycles will be modeled, as the discussion below gets a bit complicated with all of the different permutations. It might be helpful to have a model taxonomy that schematically shows how the different models are related to each other.}
%\mw{components don't need to be capitalized -- they aren't proper nouns -BW}

Modeled cycles are summarized in Table \ref{tab-cycle_sum}.

\begin{specialtable}[H] 
    \caption{Summary of all modeled non-charging and charging cycles with descriptions. \label{tab-cycle_sum}}
    \begin{tabular}{lL{0.7\linewidth}}
    \toprule
    \textbf{Cycle Label} & \textbf{Description}\\
    \midrule
    \textit{Non-Charging}\\
    C-LFR-ON & Two-cycle configuration with LFR as heat source.\\
    C-CSP-ON & Two-cycle configuration with CSP as heat source.\\
    C-1HTR1T-ON & CSP and LFR heat sources in parallel with one turbine.\\
    C-2HTR3T-ON & CSP and LFR loops each with dedicated HTR and turbine.\\
    \midrule
    \textit{Charging}\\
    C-LFR-PRE & Turbine is prior to the SALT HX.\\
    C-LFR-POST & Turbine is after the SALT HX.\\
    C-LFR-PAR & Turbine is parallel to the SALT HX.\\
    C-LFR-CIRC & Circulator bridges the LFR and SALT HX.\\
    \bottomrule
    \end{tabular}
\end{specialtable}

\subsection{Non-Charging Cycle Configurations} 
%======================================================================================

Various cycles are modeled to test their advantages and disadvantages. These cycle models fall into two categories: non-charging and charging. The non-charging category is used to determine the configuration of the cycle with a focus on electricity generation. This includes the number and location of turbines, recuperators, and heat input to the system by the CSP and LFR. To quantify the effectiveness of the non-charging configurations, a cycle efficiency, $\eta_{cycle}$, is defined in Equation \ref{eq-eta-cycle}.

\begin{equation}
    \label{eq-eta-cycle}
    \eta_{cycle} = \frac{\dot{W}_{T}-\dot{W}_{MC}-\dot{W}_{RC}}{\dot{Q}_{LFRHX}+\dot{Q}_{CSPHX}},
\end{equation}

The numerator in Equation \ref{eq-eta-cycle} is the alternator power, or the power produced from the turbines, $\dot{W}_{T}$, minus the required power of the compressors, $\dot{W}_{MC}$ and $\dot{W}_{RC}$. The denominator is the total power input into the system from the LFR HX, $\dot{Q}_{LFRHX}$, and CSP HX, $\dot{Q}_{CSPHX}$.


\subsubsection{Two-Cycle Configuration: C-LFR-ON and C-CSP-ON} %--------------------------------------------------------------------------------------

The two-cycle configuration that is tested has independent sCO$_{2}$ loops that share a common CSP salt cycle. This cycle has two sCO$_{2}$ Brayton Cycles: C-LFR-ON and C-CSP-ON. Configuration of components for these two cycles is identical with the exception of heat inputs. C-LFR-ON has heat provided from a LFR while C-CSP-ON has heat provided from the CSP. These two cycles individually operate when the focus of plant operation is primarily electricity generation. 
    
The cycle that is using the LFR heat input in the two-cycle configuration is labeled as C-LFR-ON and the cycle diagram is illustrated in Figure \ref{c-lfr-on}. 
%\mw{There are lots of little 1-2 sentence ``paragraphs'' that could be consolidated with some thought.}

\end{paracol}
\begin{figure}[H] 
    %[!h] 
    \widefigure
    \includegraphics[width=\linewidth]{Definitions/c-lfr-on.pdf}
    \caption{Diagram for C-LFR-ON with focus on electricity generation\label{c-lfr-on}}
\end{figure}
\begin{paracol}{2}
\linenumbers
\switchcolumn

Two separate sensitivity studies on the LFR inlet temperature are completed for C-LFR-ON. The constrained study is calculated by setting the LFR inlet temperature to the design value of $400^{\circ}$C, which is a requirement of the LFR primary circuit to maximize power output within material limits. In addition to the constrained studies, unconstrained studies are required to test the penalties that LFR inlet temperature has on efficiency. The unconstrained study is performed by gradually increasing the mass flow to the main compressor through a parametric study while maximizing cycle efficiency. In Figure \ref{c-lfr-on}, the location of the C2S heat exchanger while charging falls between state point 5 and 6 in any of the studied charging configurations: parallel, pre, circulator or post. 
%%\bl{flip the order of this para: constrained then unconstrained. State that the unconstrained is a 'sensitivty' case to determine the penalty of the constraint}

The cycle that is using the CSP heat input in the two-cycle configuration is labeled C-CSP-ON and the cycle diagram is shown in Figure \ref{c-csp-on}. 

\end{paracol}
\begin{figure}[H] 
    %[!h]
    \widefigure
    \includegraphics[width=\linewidth]{Definitions/c-csp-on.pdf}
    \caption{Diagram for C-CSP-ON with focus on electricity generation\label{c-csp-on}}
\end{figure}
\begin{paracol}{2}
\linenumbers
\switchcolumn

 Due to the individual operation while the cycles are generating electricity, C-CSP-ON is not directly impacted by the LFR low end temperatures. Instead, a sensitivity study is done on the temperature of the cold TES. Two temperatures are tested, $390^{\circ}$C and $440^{\circ}$C, to observe the impact of cold TES temperature on cycle efficiency. 


\subsubsection{C-1HTR1T-ON} 
%--------------------------------------------------------------------------------------

One drawback of having a two-cycle design, as seen in the C-LFR-ON and C-CSP-ON, is doubling the number of system components. Combining the two cycles into one would reduce redundancy and complexity. Heat addition from the CSP HX and LFR HX in parallel orientation is therefore studied in the C-1HTR1T-ON model. This model studies what impact mixing different temperature flows prior to the turbine has on cycle efficiency. The diagram for this cycle is illustrated in Figure \ref{c-1htr1t-on}.

%%\bl{be clear that Salt HX links back into the sCO2 cycle (e.g. parallel to 5-6) but is not shown to simplify the diagram. Otherwise people will wonder where it is coming from}
%\mw{After lots of tinkering, I've not solved how to get the figures to avoid spilling over the edge of the page. I think it's a problem with the MDPI template. The figures have a lot of whitespace between components and numbers, and you may need to try to compress some of them down to make them more compact.-BW}
\end{paracol}

\begin{figure}[H] 
    \widefigure
    \includegraphics[width=\linewidth]{Definitions/c-1htr1t-on.pdf}
    \caption{Diagram for C-1HTR1T-ON with focus on electricity generation\label{c-1htr1t-on}}
\end{figure}


\begin{paracol}{2}
\linenumbers
\switchcolumn

The C2S is located around the turbine and LFR at state points 5 to 6 depending on the charging cycle configuration: pre, parallel, post or circulator. In the C-1HTR1T-ON cycle, the LFR HX and CSP HX have identical inlet temperatures due to splitting the flow prior to their parallel orientation. Therefore, three sensitivity studies are done on the model. The initial two studies have the low LFR temperature constrained to the value of $400^{\circ}$C with varied cold CSP TES temperature and maximized cycle efficiency. To achieve a maximum cycle efficiency, the split fraction amount of flow to the main compressor, $y_{1}$, was parametrically studied.
%\mw{State what variable(s) are manipulated in order to maximize cycle efficiency -BW} 
Two cold TES temperatures are tested with constrained LFR low temperature of $400^{\circ}$C: 
\begin{itemize}
    \item	$410^{\circ}$C: Lowest cold TES temperature possible due to the sCO$_2$ cold inlet constrained from the LFR to $400^{\circ}$C and the addition of $10^{\circ}$C approach temperature;
    \item	$440^{\circ}$C: Upper bound temperature on cold TES storage;
\end{itemize}

Additionally a third test is conducted with the desired cold TES temperature of $390^{\circ}$C and the LFR low temperature unconstrained:

\begin{itemize}
    \item	$390^{\circ}$C: Unconstrained LFR cold inlet temperature allows for the desired cold TES to be achieved;
\end{itemize}

In the third case the constraint on the LFR low temperature is removed, dropping the temperature of the LFR inlet to $380^{\circ}$C. The desired cold TES temperature of $390^{\circ}$C allows for a larger temperature drop across the CSP HX increasing dispatchability.
%%\bl{I think I understand what you're talking about but the above para could spell out all the sensitivity cases a bit more clearly, perhaps as bullets. Asp you will want to explain the criteria that go into cold storage temp. Lower = more delta T = more dispatchability -BW}


\subsubsection{C-2HTR3T-ON} 
%--------------------------------------------------------------------------------------

Mixing two different temperature flows before the turbine in a Brayton cycle has a negative effect on cycle efficiency \mw{compared to...?}. To quantify the reduction in cycle efficiency, another cycle with no mixing prior to the turbine is desired. This cycle, C-2HTR3T-ON, can be seen in Figure \ref{c-2htr3t-on} and has two high temperature recuperators and three turbines. The LFR is powering one turbine, T1, and transferring unused heat to the flow entering LFR HX through a dedicated high temperature recuperator, \hl{HTR1}. 
%\mw{The LFR doesn't recuperate heat. Clarify what you mean here. -BW} 
The cycle with heat addition from the CSP also contains two separate turbines, T2, while having a dedicated high temperature recuperator, HTR2. 
\mw{Need to explain how T2 is actually 2 separate turbines represented diagramatically as 1, and why we can model it this way.} 
After the high temperature recuperators, the two flows are combined and sent to the LTR hot side. 
%\bl{this is more about having a compromise option of intermediate complexity and efficiency}

\end{paracol}
\begin{figure}[H]
    \widefigure
    \includegraphics[width=\linewidth]{Definitions/c-2htr3t-on.pdf}
    \caption{Diagram for C-2HTR3T-ON with focus on electricity generation\label{c-2htr3t-on}}
\end{figure}
\begin{paracol}{2}
\linenumbers
\switchcolumn

The C2S heat exchanger is located around the turbine and LFR at 7A to 8A depending on the charging configuration: pre, post, parallel or circulator. Three sensitivity studies are done on the C-2HTR3T-ON model -- two with the LFR low temperature constrained and one without this constraint. The two constrained studies have varied cold CSP TES temperature with the lowest temperature of $390^{\circ}$C and highest temperature of $440^{\circ}$C. The unconstrained low LFR inlet study is calculated at a cold CSP TES temperature of $390^{\circ}$C.  
%\bl{this is clearer than the previous one but bullets could still be useful}



\subsection{Thermal Energy Storage Charging Techniques} 
%======================================================================================

Charging cycle configurations accommodate energy storage modes of operation. These configurations examine the location of LFR heat extraction via the SALT HX. To maximize the available heat for extraction, alternator net power is set to zero, therefore requiring that the turbine power is equal to the compressors' demand. Despite the components being non-ideal and consuming power, the recompression cycle continues to operate, ensuring that there is mass flow to transfer heat from the Brayton cycle to C2S. The excess energy from the LFR is thermally stored in the TES for later use when grid demand increases. The location of heat extraction in the cycle is compared within the C-LFR-ON Brayton cycle by placing the salt heat exchanger in different locations around the turbine. To quantify the effectiveness of TES charging techniques, Equation \ref{eq-eta-heatstorage} defines the heat storage efficiency, $\eta_{heatstorage}$.  
%%\bl{explain that as a (re)compression cycle is still being run for the LFR and the components are not perfectly efficient, it will consume power moving the fluid around that needs to be balanced with some of the thermal output -BW}

\begin{equation}
    \label{eq-eta-heatstorage}
    \eta_{heatstorage} = \frac{\dot{Q}_{C2S}}{\dot{Q}_{LFRHX}+\dot{Q}_{CSPHX}},
\end{equation}

In the heat storage efficiency equation, $\dot{Q}_{C2S}$ is the amount of heat transferred through C2S, and the addition of $\dot{Q}_{LFRHX}$ and $\dot{Q}_{CSPHX}$ is the total amount of heat input into the system from the LFR HX and CSP HX. 
%\mw{not a full sentence -BW}

\subsubsection{C-LFR-PRE} 
%--------------------------------------------------------------------------------------

Flow leaving the turbine contains excess thermal energy that is not transformed into electrical energy. This excess thermal energy is stored in the hot CSP TES. The diagram outlining this process is C-LFR-PRE in Fig. \ref{c-lfr-pre}. 

\mw{Careful here. Be precise about what ``excess'' means in this context. Any fluid above absolute zero technically has excess thermal energy if the reference is zero. Depending on this definition, not all excess energy from the turbine is stored in TES.} 
\mw{explain in a bit more detail what is happening that makes this configuration different from prior iterations.}

\end{paracol}
\begin{figure}[H]
    \widefigure
    \includegraphics[width=\linewidth]{Definitions/c-lfr-pre.pdf}
    \caption{Diagram for C-LFR-PRE thermal energy storage charging orientation\label{c-lfr-pre}}
\end{figure}
\begin{paracol}{2}
\linenumbers
\switchcolumn

Problems arise with this salt charging configuration. The temperature out of the turbine is not high enough to charge the hot CSP TES to the required value of $595^{\circ}$C. To raise the temperature, some of the high temperature flow before the turbine is redirected through a valve and combined after the turbine. Combining different temperature flows and reducing the flow through the turbine has a large impact on heat storage efficiency. 



\subsubsection{C-LFR-POST} 
%--------------------------------------------------------------------------------------

Moving the heat extraction prior to the turbine is analyzed in C-LFR-POST. This diagram is seen in Figure \ref{c-lfr-post}.

\end{paracol}
\begin{figure}[H]
    \widefigure
    \includegraphics[width=\linewidth]{Definitions/c-lfr-post.pdf}
    \caption{Diagram for C-LFR-POST thermal energy storage charging orientation\label{c-lfr-post}}
\end{figure}
\begin{paracol}{2}
\linenumbers
\switchcolumn

This TES charging cycle extracts heat before the turbine and therefore has a large negative effect on the amount of work that the turbine is producing. The turbine power offsets the requirements of both compressors, requiring the turbine inlet temperature to be high. The amount of energy that is extracted before the turbine is small and therefore the heat storage efficiency is fractional compared to other charging techniques. There is no quantitative study done on this case because, due to the efficiency losses, it is non-viable. 

%\mw{write matter-of-factly, not in a hypothetical tense.-BW}

\subsubsection{C-LFR-PAR} 
%--------------------------------------------------------------------------------------

The requirements of the turbine and CSP hot TES can be satisfied by splitting the flow before the turbine. The flow through the salt heat exchanger in this cycle is therefore separate from the turbine. After the salt heat exchanger, a valve is needed to reduce the pressure, this TES charging cycle is C-LFR-PAR shown in Figure \ref{c-lfr-par}.

\end{paracol}
\begin{figure}[H]
    \widefigure
    \includegraphics[width=\linewidth]{Definitions/c-lfr-par.pdf}
    \caption{Diagram for C-LFR-PAR thermal energy storage charging orientation\label{c-lfr-par}}
\end{figure}
\begin{paracol}{2}
\linenumbers
\switchcolumn

Two sensitivity studies \mw{this is one study with two parameter values, right?} with varying cold CSP TES temperature are carried out to determine the impact on heat storage efficiency. The study considers two temperature values of $390^{\circ}$C and $440^{\circ}$C.

\subsubsection{C-LFR-CIRC} 
%--------------------------------------------------------------------------------------

The full diagram for C-LFR-CIRC is shown in Figure \ref{c-lfr-circ}.

\end{paracol}
\begin{figure}[H]
    \widefigure
    \includegraphics[width=\linewidth]{Definitions/c-lfr-circ.pdf}
    \caption{Full diagram for C-LFR-CIRC thermal energy storage charging orientation\label{c-lfr-circ}}
\end{figure}
\begin{paracol}{2}
\linenumbers
\switchcolumn

The charging subsection of this diagram is composed of a circulation cycle that has heat inputted through the LFR heat exchanger. A separated circulation cycle has a loop which avoids the losses associated with compressor and turbine, therefore achieving higher heat storage efficiency than possible with full cycle operation. This subsection is encircled in blue and can be seen in Figure \ref{c-lfr-circ-sub}.

%%\bl{explain why you are considering this case (avoid losing useful energy in compressor and turbine inefficiencies)} \mw{agree with Ben. Additional motivation is needed -BW}

\end{paracol}
\begin{figure}[H]
    \widefigure
    \includegraphics[width=8 cm]{Definitions/c-lfr-circ-sub.pdf}
    \caption{Diagram for C-LFR-CIRC sub-cycle thermal energy storage charging orientation\label{c-lfr-circ-sub}}
\end{figure}
\begin{paracol}{2}
\linenumbers
\switchcolumn

The flow continues through a circulator which is assumed to have negligible pressure rise (i.e. there is assumed to be negligible pressure drop in this case). A heat exchanger, SALT HX, extracts heat from the flow, storing the thermal energy in the hot TES for later use. Excess heat that is not extracted is then dumped into a reservoir through the chiller to bring the temperature of the flow down to LFR cool side operating temperature of $400^{\circ}$C. Three different cold TES temperatures; $390^{\circ}$C, $410^{\circ}$C, and $440^{\circ}$C, are compared in a sensitivity study. 



%%%%%%%%%%%%%%%%%%%%%%%%%%%%%%%%%%%%%%%%%%%%%%%%%%%%%%%%%%%%%%%%%%%%%%%%%%%%%%%%%%%%%%%
\section{Results and Discussion}

For all cycle configurations presented, with constrained or unconstrained LFR low temperature inlet and varied cold TES temperature, cycle and heat storage efficiencies were maximized using parametric studies on flow splitter fractions. These results were obtained using standardized values found in Table \ref{tab-cycle_sum} for a more direct comparison between cycles. Presentation and discussion of the results is fulfilled in the following section. 

\subsection{Non-Charging Cycle Configurations}
%======================================================================================

The values for the three non-charging cycle models with a focus on electrical generation, 2-cycle configuration composed of C-LFR-ON and C-CSP-ON, C-1HTR1T-ON, and C-2HTR3T-ON, are displayed in Table \ref{tab-noncharg}. 

\end{paracol}
\begin{specialtable}[H] 
    %[htbp]
    \caption{Calculated system parameters for non-charging cycle configurations with constrained (\textit{C}) and unconstrained (\textit{U}) lead-cooled fast reactor low-end temperature.\label{tab-noncharg}}
    \resizebox{\columnwidth}{!}{
    \begin{tabular}{lccccccccccc}
    \toprule
     &  & \textbf{C-LFR-ON} & & \textbf{C-CSP-ON} &  &	\textbf{C-1HTR1T-ON} &	&	&	\textbf{C-2HTR3T-ON}	&	& \\
    \textbf{Definition} & \textbf{Variable} & \textit{U} & \textit{C} & \textit{N/A} & \textit{N/A} &	\textit{U}	&	\textit{C}	&	\textit{C}	&	\textit{U}	&	\textit{C}	&	\textit{C}
    \\
    \midrule
    Cold TES Temperature ($^{\circ}$C)	&	$T_{CS}$	&	N/A	&	N/A	&	390	&	440	&	390	&	410	&	440	&	390	&	390	&	440	\\
    LFR Inlet Temperature ($^{\circ}$C)	&	$T_{4,4C,5A}$	&	415.1	&	400	&	N/A	&	N/A	&	380	&	400	&	400	&	415.2	&	400	&	400	\\
    Cycle Efficiency (\%)	&	$\eta_{cycle}$	&	47.08	&	45.28	&	45.58	&	45.93	&	44.9	&	44.22	&	44.22	&	46.10	&	44.34	&	44.35	\\
    Alternator Power (MW)	&	$\dot{W}_{A}$	&	447.3	&	430.2	&	339.1	&	340.6	&	763.9	&	752.3	&	752.6	&	783.7	&	753.7	&	754	\\
    PC Heat Transfer (MW)	&	$\dot{Q}_{PC}$	&	502.7	&	519.8	&	418.5	&	420.2	&	937.4	&	949.2	&	949.3	&	917.6	&	947.6	&	947.9	\\
    MC Power (MW)	&	$\dot{W}_{MC}$	&	115	&	118.9	&	95.73	&	96.13	&	214.5	&	194.8	&	194.8	&	209.9	&	216.8	&	216.9	\\
    RC Power (MW)	&	$\dot{W}_{RC}$	&	116.6	&	77.3	&	97.22	&	97.71	&	217.4	&	285.7	&	285.7	&	213.5	&	140.8	&	140.8	\\
    T1 Power (MW)	&	$\dot{W}_{T1}$	&	678.9	&	626.4	&	532.1	&	534.5	&	1196	&	1233	&	1233	&	679.3	&	626.3	&	626.3	\\
    T2 Power (MW)	&	$\dot{W}_{T2}$	&	N/A	&	N/A	&	N/A	&	N/A	&	N/A	&	N/A	&	N/A	&	527.8	&	485	&	485.4	\\
    MC Mass Flow Fraction (-)	&	$y_{1}$	&	0.7	&	0.7844	&	0.6996	&	0.6994	&	0.7	&	0.6333	&	0.6333	&	0.6993	&	0.7846	&	0.7846	\\
    LFR Mass Flow Fraction (-)	&	$y_{2}$	&	N/A	&	N/A	&	N/A	&	N/A	&	0.4485	&	0.4928	&	0.4929	&	0.5478	&	0.5486	&	0.5484	\\
    LTR UA Value (MW/$^{\circ}$C)	&	$UA_{LTR}$	&	54.68	&	22.84	&	45.4	&	45.53	&	102	&	91.63	&	91.64	&	134.4	&	41.56	&	41.57	\\
    LTR Capacitance Ratio (-)	&	$CR_{LTR}$	&	0.9867	&	0.8473	&	0.9858	&	0.9854	&	0.9867	&	0.9066	&	0.9066	&	0.9853	&	0.847	&	0.847	\\
    LTR Heat Transfer Rate (MW)	&	$\dot{Q}_{LTR}$	&	656.8	&	366.5	&	548.3	&	551.4	&	549.2	&	614.9	&	615.1	&	1204	&	667.1	&	667.3	\\
    LTR Effectiveness (-)	&	$\varepsilon_{LTR}$	&	0.92	&	0.8742	&	0.9201	&	0.9202	&	0.92	&	0.9485	&	0.9485	&	0.9414	&	0.8741	&	0.8741	\\
    HTR UA Value (MW/$^{\circ}$C)	&	$UA_{HTR}$	&	48.29	&	42.71	&	34.58	&	34.72	&	78.22	&	82.3	&	82.34	&	48.32	&	42.69	&	42.69	\\
    HTR Capacitance Ratio (-)	&	$CR_{HTR}$	&	0.8657	&	0.8143	&	0.8593	&	0.8594	&	0.8595	&	0.8754	&	0.8755	&	0.8661	&	0.8142	&	0.8142	\\
    HTR Heat Transfer Rate (MW)	&	$\dot{Q}_{HTR}$	&	998.1	&	1161	&	665.4	&	667.7	&	679.2	&	742.6	&	743	&	545.7	&	636.8	&	636.6	\\
    HTR Effectiveness (-)	&	$\varepsilon_{HTR}$	&	0.9544	&	0.9627	&	0.9436	&	0.9436	&	0.9445	&	0.9441	&	0.9441	&	0.9542	&	0.9627	&	0.9627	\\
    HTR2 UA Value (MW/$^{\circ}$C)	&	$UA_{HTR2}$	&	N/A	&	N/A	&	N/A	&	N/A	&	N/A	&	N/A	&	N/A	&	34.29	&	31.61	&	31.63	\\
    HTR2 Capacitance Ratio (-)	&	$CR_{HTR2}$	&	N/A	&	N/A	&	N/A	&	N/A	&	N/A	&	N/A	&	N/A	&	0.8594	&	0.8074	&	0.8074	\\
    HTR2 Heat Transfer Rate (MW)	&	$\dot{Q}_{HTR2}$	&	N/A	&	N/A	&	N/A	&	N/A	&	N/A	&	N/A	&	N/A	&	298.1	&	363.4	&	363.8	\\
    HTR2 Effectiveness (-)	&	$\varepsilon_{HTR2}$	&	N/A	&	N/A	&	N/A	&	N/A	&	N/A	&	N/A	&	N/A	&	0.9436	&	0.9561	&	0.9561	\\
    CSPHX UA Value (MW/$^{\circ}$C)	&	$UA_{CSPHX}$	&	N/A	&	N/A	&	71.34	&	26.92	&	33.72	&	73.4	&	35.05	&	70.88	&	44.92	&	23.13	\\
    CSPHX Capacitance Ratio (-)	&	$CR_{CSPHX}$	&	N/A	&	N/A	&	0.9924	&	0.701	&	0.8104	&	0.9957	&	0.8034	&	0.9926	&	0.9138	&	0.6454	\\
    CSPHX Heat Transfer Rate (MW)	&	$\dot{Q}_{CSPHX}$	&	N/A	&	N/A	&	757.6	&	760.8	&	751.3	&	751.5	&	751.9	&	751.3	&	751.3	&	751.9	\\
    CSPHX Effectiveness (-)	&	$\varepsilon_{CSPHX}$	&	N/A	&	N/A	&	0.945	&	0.945	&	0.945	&	0.9381	&	0.9374	&	0.9450	&	0.9493	&	0.9493	\\

    \bottomrule
    \end{tabular}
    }
\end{specialtable}
\begin{paracol}{2}
\linenumbers
\switchcolumn

The calculated values reported in Table \ref{tab-noncharg} are component power, mass flow fractions, and counter-flow heat exchanger specifications. Columns labeled with '$C$' headers have the LFR sCO$_2$ low inlet temperature constrained to 400$^{\circ}$C while those labeled with '$U$' are unconstrained and calculated based off of a parametric study on the MC mass flow fraction. Values listed in the Cold TES Temperature row are set to values of 390$^{\circ}$C, 410$^{\circ}$C, or 440$^{\circ}$C to study the effects that cold TES temperature has on cycle efficiency. Cycles with data entries labeled not applicable (N/A) do not contain the component listed and therefore do not have values associated with the component. 

\subsubsection{Two-Cycle Configuration: C-LFR-ON and C-CSP-ON}

The two-cycle configuration has the CSP and LFR operating in separate recompression cycles when the focus is on electrical generation, therefore these two cycles are analyzed individually. C-LFR-ON has the highest efficiency during unconstrained operation with a gain of 1.8\%. The unconstrained case allows for smaller temperature gradient across the LFR, an increase in recuperator effectiveness, and an increase in recuperator capacitance ratio, all of which reduce sources of irreversibility in the cycle \cite{klein_nellis_2011}. According to studies carried out by our industry partner, Westinghouse, increasing the LFR sCO$_2$ inlet temperature above the design value of 400$^{\circ}$C has a negative effect on LFR HX outlet temperature, effectively demanding an ideal LFR heat exchanger. Additionally, Westinghouse concluded that a lower sCO$_2$, down to 340$^{\circ}$C, is desirable for LFR heat generation efficiency. The sCO$_2$ inlet temperature to the LFR increases by 15.1$^{\circ}$C to a value of 415.1$^{\circ}$C, reducing LFR heat generation efficiency and demanding a larger, more idealized, LFR heat exchanger.   
\hl{would I need more information on the LFR from westinghouse (ie report) to make these statements? additionally a conclusion to these statements needs more supporting detail, should I go with the higher efficiency or the cycle that I know works with design values?}

C-CSP-ON is not affected by constrained or unconstrained LFR sCO$_2$ inlet temperature during non-charging configuration because the two cycles are not tethered. Two individual studies were conducted on the C-CSP-ON cycle cold TES temperature, one with a lower design value of 390$^{\circ}$C and another with a higher value of 440$^{\circ}$C. A cold TES temperature of 410$^{\circ}$C was excluded since it can be presumed to have an efficiency intermediate between the 390$^{\circ}$C and 440$^{\circ}$C calculated values. Cold TES temperature was found to have a negligible effect on cycle efficiency between the two studied cases, with the 440$^{\circ}$C case gaining 0.35\%. The component that has the largest change in values is the CSP HX with a large drop in capacitance ratio and UA value, therefore the lower temperature of 390$^{\circ}$C for cold TES temperature is ideal increasing the performances of the heat exchanger. 


\subsubsection{C-1HTR1T-ON}
The single cycle configuration, C-1HTR1T-ON, has the CSP and LFR heat additions in parallel causing identical inlet conditions. Identical inlet conditions lead to a lower bound of 410$^{\circ}$C on the cold TES temperature when the sCO$_2$ LFR inlet temperature is constrained to 400$^{\circ}$C. A supplementary 440$^{\circ}$C cold TES study is ran to test the effects that higher temperature storage has on cycle efficiency. Raising the temperature of the cold TES from 410$^{\circ}$C to 440$^{\circ}$C had no effect on cycle efficiency but did raise the solar salt mass flow rate to accommodate for the lower temperature difference across the CSP HX hot side. Increasing the solar salt mass flow rate requires more storage for the same heat input and larger pumps increasing the cost of the system. Therefore, a lower cold TES temperature is desirable. Without the constraint on the sCO$_2$ LFR inlet temperature, a study is additionally conducted with cold TES temperature set to 390$^{\circ}$C. Of these three sensitivity studies, the cycle that yielded the highest efficiency is the unconstrained, 390$^{\circ}$C cold TES with an efficiency gain of 0.68\% over the constrained cases. The high temperature difference between the hot and cold TES reduces the mass flow rate of solar salt and subsequent cost of system components. The sCO$_2$ LFR inlet temperature for the 390$^{\circ}$C cold TES study is 380$^{\circ}$C, which increases the performance of the LFR heat exchanger as well as allowing for a higher LFR efficiency. 

\subsubsection{C-2HTR3T-ON}
C-2HTR3T-ON has dedicated recuperators and turbines for the LFR and CSP, allowing for independent inlet temperatures. Three studies were conducted on this cycle with two being constrained and one unconstrained sCO$_2$ LFR inlet conditions. The constrained studies have cold TES temperatures of 390$^{\circ}$C and 440$^{\circ}$, both of which have negligible efficiency differences. Similarly to the studies done on C-1HTR1T-ON, the cold TES temperature had an effect on mass flow rate of solar salt with the 440$^{\circ}$C case having a larger value and therefore higher component cost. The 390$^{\circ}$C unconstrained study has the highest efficiency with a value of 46.10\%, a 1.76\% increase over the constrained cases. Additionally, the sCO$_2$ LFR inlet temperature increased by a value of 15.2$^{\circ}$C to 415.2$^{\circ}$C, having a negative effect on LFR efficiency and heat exchanger parameters. 

\hl{again how do I justify the heat exchanger and LFR efficiency gains? I do not have any sources other than what westinghouse has supplied over email and in general meetings}

\subsection{Thermal Energy Storage Charging Techniques}

When the focus of the cycles is thermal storage for later use, the LFR charges the hot TES through the C2S heat exchanger. Studies done on the charging techniques, C-LFR-PRE, C-LFR-PAR, and C-LFR-CIRC, have constrained sCO$_2$ LFR inlet temperatures to 400$^{\circ}$C with set values of cold TES temperature of 390$^{\circ}$C, 410$^{\circ}$C, or 440$^{\circ}$C. The results are presented in Table \ref{tab-charg}. 

\end{paracol}
\begin{specialtable}[H] 
    %[htbp]
    \caption{Calculated system parameters for charging cycle configurations with constrained (\textit{C}) and unconstrained (\textit{U}) lead-cooled fast reactor low-end temperature.\label{tab-charg}}
    \resizebox{\columnwidth}{!}{
    \begin{tabular}{lccccccc}
    \toprule
     &  & \textbf{C-LFR-PRE} & \textbf{C-LFR-PAR} &  &	\textbf{C-LFR-CIRC} &	& \\
    \textbf{Definition} & \textbf{Variable} & \textit{C} & \textit{C} & \textit{C} & \textit{C} &	\textit{C}	&	\textit{C}\\
    \midrule
    Cold TES Temperature ($^{\circ}$C)	&	$T_{CS}$	&	390	&	390	&	440	&	390	&	410	&	440	\\
    LFR Inlet Temperature ($^{\circ}$C)	&	$T_{4,1C}$	&	400	&	400	&	400	&	400	&	400	&	400	\\
    Heat Storage Efficiency (\%)	&	$\eta_{heatstorage}$	&	34.53	&	45.30	&	45.30	&	99.92	&	89.66	&	74.29	\\
    Alternator Power (MW)	&	$\dot{W}_{A}$	&	0	&	0	&	0	&	N/A	&	N/A	&	N/A	\\
    PC Heat Transfer (MW)	&	$\dot{Q}_{PC}$	&	622	&	519.6	&	519.6	&	N/A	&	N/A	&	N/A	\\
    MC Power (MW)	&	$\dot{W}_{MC}$	&	142.3	&	118.9	&	118.9	&	N/A	&	N/A	&	N/A	\\
    RC Power (MW)	&	$\dot{W}_{RC}$	&	21.89	&	77.28	&	77.28	&	N/A	&	N/A	&	N/A	\\
    Turbine Power (MW)	&	$\dot{W}_{T}$	&	164.2	&	196.2	&	196.2	&	N/A	&	N/A	&	N/A	\\
    Chiller Heat Transfer (MW)	&	$\dot{Q}_{chill}$	&	N/A	&	N/A	&	N/A	&	0.7245	&	98.25	&	244.2	\\
    MC Mass Flow Fraction (-)	&	$y_{1}$	&	0.9389	&	0.7844	&	0.7844	&	N/A	&	N/A	&	N/A	\\
    SALT HX Mass Flow Fraction (-)	&	$y_{2}$	&	N/A	&	0.6867	&	0.6867	&	N/A	&	N/A	&	N/A	\\
    Valve Mass Flow Fraction (-)	&	$y_{5}$	&	0.7378	&	N/A	&	N/A	&	N/A	&	N/A	&	N/A	\\
    LTR UA Value (MW/$^{\circ}$C)	&	$UA_{LTR}$	&	6.99	&	22.83	&	22.83	&	N/A	&	N/A	&	N/A	\\
    LTR Capacitance Ratio (-)	&	$CR_{LTR}$	&	0.6754	&	0.8473	&	0.8743	&	N/A	&	N/A	&	N/A	\\
    LTR Heat Transfer Rate (MW)	&	$\dot{Q}_{LTR}$	&	93.05	&	366.4	&	366.4	&	N/A	&	N/A	&	N/A	\\
    LTR Effectiveness (-)	&	$\varepsilon_{LTR}$	&	0.6384	&	0.8742	&	0.8742	&	N/A	&	N/A	&	N/A	\\
    HTR UA Value (MW/$^{\circ}$C)	&	$UA_{HTR}$	&	41.69	&	42.69	&	42.69	&	N/A	&	N/A	&	N/A	\\
    HTR Capacitance Ratio (-)	&	$CR_{HTR}$	&	0.7531	&	0.8143	&	0.8143	&	N/A	&	N/A	&	N/A	\\
    HTR Heat Transfer Rate (MW)	&	$\dot{Q}_{HTR}$	&	1567	&	1160	&	1160	&	N/A	&	N/A	&	N/A	\\
    HTR Effectiveness (-)	&	$\varepsilon_{HTR}$	&	0.9695	&	0.9627	&	0.9627	&	N/A	&	N/A	&	N/A	\\
    CSPHX UA Value (MW/$^{\circ}$C)	&	$UA_{CSPHX}$	&	9.04	&	8.037	&	14.24	&	48.29	&	43.16	&	35.63	\\
    CSPHX Capacitance Ratio (-)	&	$CR_{CSPHX}$	&	0.4735	&	0.7556	&	0.9339	&	0.8755	&	0.8599	&	0.8294	\\
    CSPHX Heat Transfer Rate (MW)	&	$\dot{Q}_{CSPHX}$	&	328	&	430.4	&	430.4	&	949.3	&	851.8	&	705.8	\\
    CSPHX Effectiveness (-)	&	$\varepsilon_{CSPHX}$	&	0.9368	&	0.8275	&	0.8307	&	0.9511	&	0.9459	&	0.9355	\\
    CSPHX Approach Temperature ($^{\circ}$C)	&	$\delta_{CSPHX}$	&	N/A	&	35	&	26.27	&	N/A	&	N/A	&	N/A	\\
    \bottomrule
    \end{tabular}
    }
\end{specialtable}
\begin{paracol}{2}
\linenumbers
\switchcolumn

\subsubsection{C-LFR-PRE}



\subsubsection{C-LFR-PAR}


\subsubsection{C-LFR-CIRC}


% \subsection{Small Tables}
% %--------------------------------------------------------------------------------------

% Modeling the C-LFR-ON cycle in EES yielded the results in Table \ref{tab-c-lfr-on}. 

% \begin{specialtable}[H] 
%     %[htbp]
%     \caption{Calculated system parameters for non-charging C-LFR-ON cycle configuration with constrained (\textit{C}) and unconstrained (\textit{U}) lead-cooled fast reactor low-end temperature.\label{tab-c-lfr-on}}
%     \begin{tabular}{lccc}
%     \toprule
%     \textbf{Definition} & \textbf{Variable} & \textit{U} & \textit{C}\\
%     \midrule
%     LFR Inlet Temperature ($^{\circ}$C)	&	$T_{4}$	&	395.1	&	400	\\
%     Cycle Efficiency (\%)	&	$\eta_{cycle}$	&	47.08	&	45.28	\\
%     Alternator Power (MW)	&	$\dot{W}_{A}$	&	447.3	&	430.2	\\
%     PC Heat Transfer (MW)	&	$\dot{Q}_{PC}$	&	502.7	&	519.8	\\
%     MC Power (MW)	&	$\dot{W}_{MC}$	&	115	&	118.9	\\
%     RC Power (MW)	&	$\dot{W}_{RC}$	&	116.6	&	77.3	\\
%     Turbine Power (MW)	&	$\dot{W}_{T}$	&	678.9	&	626.4	\\
%     MC Mass Flow Fraction (-)	&	$y_{1}$	&	0.7	&	0.7844	\\
%     LTR UA Value (MW/$^{\circ}$C)	&	$UA_{LTR}$	&	54.68	&	22.84	\\
%     LTR Capacitance Ratio (-)	&	$CR_{LTR}$	&	0.9867	&	0.8473	\\
%     LTR Heat Transfer Rate (MW)	&	$\dot{Q}_{LTR}$	&	656.8	&	366.5	\\
%     LTR Effectiveness (-)	&	$\varepsilon_{LTR}$	&	0.92	&	0.8742	\\
%     HTR UA Value (MW/$^{\circ}$C)	&	$UA_{HTR}$	&	48.29	&	42.71	\\
%     HTR Capacitance Ratio (-)	&	$CR_{HTR}$	&	0.8657	&	0.9627	\\
%     HTR Heat Transfer Rate (MW)	&	$\dot{Q}_{HTR}$	&	998.1	&	1161	\\
%     HTR Effectiveness (-)	&	$\varepsilon_{HTR}$	&	0.9544	&	0.9627	\\
%     \bottomrule
%     \end{tabular}\\
% \end{specialtable}

% \textbf{Discussion of Results}



% The EES model outputs for C-CSP-ON are listed in Table \ref{tab-c-csp-on}.

% \begin{specialtable}[H] 
%     %[htbp] 
%     \caption{Calculated system parameters for non-charging C-CSP-ON cycle configuration with varied TES cold temperature. \label{tab-c-csp-on}}
%     \begin{tabular}{lccc}
%     \toprule
%     \textbf{Definition} & \textbf{Variable} &  &\\
%     \midrule
%     Cold TES Temperature ($^{\circ}$C)	&	$T_{CS}$	&	390	&	440	\\
%     Cycle Efficiency (\%)	&	$\eta_{cycle}$	&	45.58	&	45.93	\\
%     Alternator Power (MW)	&	$\dot{W}_{A}$	&	339.1	&	340.6	\\
%     PC Heat Transfer (MW)	&	$\dot{Q}_{PC}$	&	418.5	&	420.2	\\
%     MC Power (MW)	&	$\dot{W}_{MC}$	&	95.73	&	96.13	\\
%     RC Power (MW)	&	$\dot{W}_{RC}$	&	97.22	&	97.71	\\
%     Turbine Power (MW)	&	$\dot{W}_{T}$	&	532.1	&	534.5	\\
%     MC Mass Flow Fraction (-)	&	$y_{1}$	&	0.6996	&	0.6994	\\
%     LTR UA Value (MW/$^{circ}$C)	&	$UA_{LTR}$	&	45.4	&	45.53	\\
%     LTR Capacitance Ratio (-)	&	$CR_{LTR}$	&	0.9858	&	0.9854	\\
%     LTR Heat Transfer Rate (MW)	&	$\dot{Q}_{LTR}$	&	548.3	&	551.4	\\
%     LTR Effectiveness (-)	&	$\varepsilon_{LTR}$	&	0.9201	&	0.9202	\\
%     HTR UA Value (MW/$^{\circ}$C)	&	$UA_{HTR}$	&	34.58	&	34.72	\\
%     HTR Capacitance Ratio (-)	&	$CR_{HTR}$	&	0.8593	&	0.8594	\\
%     HTR Heat Transfer Rate (MW)	&	$\dot{Q}_{HTR}$	&	665.4	&	667.7	\\
%     HTR Effectiveness (-)	&	$\varepsilon_{HTR}$	&	0.9436	&	0.9436	\\
%     CSPHX UA Value (MW/$^{\circ}$C)	&	$UA_{CSPHX}$	&	71.34	&	26.92	\\
%     CSPHX Capacitance Ratio (-)	&	$CR_{CSPHX}$	&	0.9924	&	0.701	\\
%     CSPHX Heat Transfer Rate (MW)	&	$\dot{Q}_{CSPHX}$	&	757.6	&	760.8	\\
%     CSPHX Effectiveness (-)	&	$\varepsilon_{CSPHX}$	&	0.945	&	0.945	\\
%     \bottomrule
%     \end{tabular}\\
% \end{specialtable}

% \textbf{Discussion of Results}

% %--------------------------------------------------------------------------------------

% These values are displayed in Table \ref{tab-c-1htr1t-on}.

% \begin{specialtable}[H] 
%     %[htbp] 
%     \caption{Calculated system parameters for non-charging C-1HTR1T-ON cycle configuration with constrained (\textit{C}) and unconstrained (\textit{U}) lead-cooled fast reactor low-end temperature. Temperature of TES cold temperature is also varied.\label{tab-c-1htr1t-on}}
%     \begin{tabular}{lcccc}
%     \toprule
%     \textbf{Definition} & \textbf{Variable} & \textbf{C-1HTR1T-ON}\\
%     & & \textit{U} & \textit{C} & \textit{C}\\
%     \midrule	
%     Cold TES Temperature ($^{\circ}$C)	&	$T_{CS}$	&	390	&	410	&	440	\\
%     LFR Inlet Temperature ($^{\circ}$C)	&	$T_{4C}$	&	380	&	400	&	400	\\
%     Cycle Efficiency (\%)	&	$\eta_{cycle}$	&	44.9	&	44.22	&	44.22	\\
%     Alternator Power (MW)	&	$\dot{W}_{A}$	&	763.9	&	752.3	&	752.6	\\
%     PC Heat Transfer (MW)	&	$\dot{Q}_{PC}$	&	937.4	&	949.2	&	949.3	\\
%     MC Power (MW)	&	$\dot{W}_{MC}$	&	214.5	&	194.8	&	194.8	\\
%     RC Power (MW)	&	$\dot{W}_{RC}$	&	217.4	&	285.7	&	285.7	\\
%     Turbine Power (MW)	&	$\dot{W}_{T}$	&	1196	&	1233	&	1233	\\
%     MC Mass Flow Fraction (-)	&	$y_{1}$	&	0.7	&	0.6333	&	0.6333	\\
%     LFR Mass Flow Fraction (-)	&	$y_{2}$	&	0.4485	&	0.4928	&	0.4929	\\
%     LTR UA Value (MW/$^{\circ}$C)	&	$UA_{LTR}$	&	102	&	91.63	&	91.64	\\
%     LTR Capacitance Ratio (-)	&	$CR_{LTR}$	&	0.9867	&	0.9066	&	0.9066	\\
%     LTR Heat Transfer Rate (MW)	&	$\dot{Q}_{LTR}$	&	549.2	&	614.9	&	615.1	\\
%     LTR Effectiveness (-)	&	$\varepsilon_{LTR}$	&	0.92	&	0.9485	&	0.9485	\\
%     HTR UA Value (MW/$^{\circ}$C)	&	$UA_{HTR}$	&	78.22	&	82.3	&	82.34	\\
%     HTR Capacitance Ratio (-)	&	$CR_{HTR}$	&	0.8595	&	0.8754	&	0.8755	\\
%     HTR Heat Transfer Rate (MW)	&	$\dot{Q}_{HTR}$	&	679.2	&	742.6	&	743	\\
%     HTR Effectiveness (-)	&	$\varepsilon_{HTR}$	&	0.9445	&	0.9441	&	0.9441	\\
%     CSPHX UA Value (MW/$^{\circ}$C)	&	$UA_{CSPHX}$	&	33.72	&	73.4	&	75.19	\\
%     CSPHX Capacitance Ratio (-)	&	$CR_{CSPHX}$	&	0.8104	&	0.9957	&	0.8034	\\
%     CSPHX Heat Transfer Rate (MW)	&	$\dot{Q}_{CSPHX}$	&	751.3	&	751.5	&	751.9	\\
%     CSPHX Effectiveness (-)	&	$\varepsilon_{CSPHX}$	&	0.945	&	0.9381	&	0.9374	\\
%     \bottomrule
%     \end{tabular}\\
% \end{specialtable}

% \textbf{Discussion of Results}

% %--------------------------------------------------------------------------------------

% The calculated values from these studies are displayed in Table \ref{tab-c-2htr3t-on}.

% \begin{specialtable}[H]
%     %[htbp]
%     \caption{Calculated system parameters for non-charging C-2HTR3T-ON cycle configuration with constrained (\textit{C}) and unconstrained (\textit{U}) lead-cooled fast reactor low-end temperature.\label{tab-c-2htr3t-on}}
%     \begin{tabular}{lcccc}
%     \toprule
%     \textbf{Definition} & \textbf{Variable} & \textbf{C-2HTR3T-ON} &\\
%     & & \textit{U} & \textit{C} & \textit{C}\\
%     \midrule	
%     Cold TES Temperature ($^{\circ}$C)	&	$T_{CS}$	&	390	&	390	&	440	\\
%     LFR Inlet Temperature ($^{\circ}$C)	&	$T_{5A}$	&	395.2	&	400	&	400	\\
%     Cycle Efficiency (\%)	&	$\eta_{cycle}$	&	46.10	&	44.34	&	44.35	\\
%     Alternator Power (MW)	&	$\dot{W}_{A}$	&	783.7	&	753.7	&	754	\\
%     PC Heat Transfer (MW)	&	$\dot{Q}_{PC}$	&	917.6	&	947.6	&	947.9	\\
%     MC Power (MW)	&	$\dot{W}_{MC}$	&	209.9	&	216.8	&	216.9	\\
%     RC Power (MW)	&	$\dot{W}_{RC}$	&	213.5	&	140.8	&	140.8	\\
%     T1 Power (MW)	&	$\dot{W}_{T1}$	&	679.3	&	626.3	&	626.3	\\
%     T2 Power (MW)	&	$\dot{W}_{T2}$	&	527.8	&	485	&	485.4	\\
%     MC Mass Flow Fraction (-)	&	$y_{1}$	&	0.6993	&	0.7846	&	0.7846	\\
%     LFR Mass Flow Fraction (-)	&	$y_{2}$	&	0.5478	&	0.5486	&	5.48E-01	\\
%     LTR UA Value (MW/$^{\circ}$C)	&	$UA_{LTR}$	&	134.4	&	41.56	&	41.57	\\
%     LTR Capacitance Ratio (-)	&	$CR_{LTR}$	&	0.9853	&	0.847	&	0.847	\\
%     LTR Heat Transfer Rate (MW)	&	$\dot{Q}_{LTR}$	&	1204	&	667.1	&	667.3	\\
%     LTR Effectiveness (-)	&	$\varepsilon_{LTR}$	&	0.9414	&	0.8741	&	0.8741	\\
%     HTR1 UA Value (MW/$^{\circ}$C)	&	$UA_{HTR1}$	&	48.32	&	42.69	&	42.69	\\
%     HTR1 Capacitance Ratio (-)	&	$CR_{HTR1}$	&	0.8661	&	0.8142	&	0.8142	\\
%     HTR1 Heat Transfer Rate (MW)	&	$\dot{Q}_{HTR1}$	&	545.7	&	636.8	&	636.6	\\
%     HTR1 Effectiveness (-)	&	$\varepsilon_{HTR1}$	&	0.9542	&	0.9627	&	0.9627	\\
%     HTR2 UA Value (MW/$^{\circ}$C)	&	$UA_{HTR2}$	&	34.29	&	31.61	&	31.63	\\
%     HTR2 Capacitance Ratio (-)	&	$CR_{HTR2}$	&	0.8594	&	0.8074	&	0.8074	\\
%     HTR2 Heat Transfer Rate (MW)	&	$\dot{Q}_{HTR2}$	&	298.1	&	363.4	&	363.8	\\
%     HTR2 Effectiveness (-)	&	$\varepsilon_{HTR2}$	&	0.9436	&	0.9561	&	0.9561	\\
%     CSPHX UA Value (MW/$^{\circ}$C)	&	$UA_{CSPHX}$	&	70.88	&	44.92	&	23.13	\\
%     CSPHX Capacitance Ratio (-)	&	$CR_{CSPHX}$	&	0.9926	&	0.9138	&	0.6454	\\
%     CSPHX Heat Transfer Rate (MW)	&	$\dot{Q}_{CSPHX}$	&	751.3	&	751.3	&	751.9	\\
%     CSPHX Effectiveness (-)	&	$\varepsilon_{CSPHX}$	&	0.9450	&	0.9493	&	0.9493	\\
%     \bottomrule
%     \end{tabular}\\
% \end{specialtable}

% \textbf{Discussion of Results}

% %======================================================================================

% %--------------------------------------------------------------------------------------

% The calculations from this TES charging technique are shown in Table \ref{tab-c-lfr-pre}.

% \begin{specialtable}[H]
%     %[htbp]
%     \caption{Calculated system parameters for salt charging C-LFR-PRE cycle configuration with TES cold storage set to 663.2 K.\label{tab-c-lfr-pre}}
%     \begin{tabular}{lcc}
%     \toprule
%     \textbf{Definition} & \textbf{Variable} & \textbf{C-LFR-PRE}\\
%     & & \textit{C}\\
%     \midrule	
%     Cold TES Temperature ($^{\circ}$C)	&	$T_{CS}$	&	390	\\
%     LFR Inlet Temperature ($^{\circ}$C)	&	$T_{4}$	&	400	\\
%     Heat Storage Efficiency (\%)	&	$\eta_{heatstorage}$	&	34.53	\\
%     Alternator Power (MW)	&	$\dot{W}_{A}$	&	0	\\
%     PC Heat Transfer (MW)	&	$\dot{Q}_{PC}$	&	622	\\
%     MC Power (MW)	&	$\dot{W}_{MC}$	&	142.3	\\
%     RC Power (MW)	&	$\dot{W}_{RC}$	&	21.89	\\
%     Turbine Power (MW)	&	$\dot{W}_{T}$	&	164.2	\\
%     MC Mass Flow Fraction (-)	&	$y_{1}$	&	0.9389	\\
%     Valve Mass Flow Fraction (-)	&	$y_{5}$	&	0.7378	\\
%     LTR UA Value (MW/$^{\circ}$C)	&	$UA_{LTR}$	&	0.699	\\
%     LTR Capacitance Ratio (-)	&	$CR_{LTR}$	&	0.6754	\\
%     LTR Heat Transfer Rate (MW)	&	$\dot{Q}_{LTR}$	&	93.05	\\
%     LTR Effectiveness (-)	&	$\varepsilon_{LTR}$	&	0.6384	\\
%     HTR UA Value (MW/$^{\circ}$C)	&	$UA_{HTR}$	&	41.69	\\
%     HTR Capacitance Ratio (-)	&	$CR_{HTR}$	&	0.7531	\\
%     HTR Heat Transfer Rate (MW)	&	$\dot{Q}_{HTR}$	&	1567	\\
%     HTR Effectiveness (-)	&	$\varepsilon_{HTR}$	&	0.9695	\\
%     CSPHX UA Value (MW/$^{\circ}$C)	&	$UA_{CSPHX}$	&	0.904	\\
%     CSPHX Capacitance Ratio (-)	&	$CR_{CSPHX}$	&	0.4735	\\
%     CSPHX Heat Transfer Rate (MW)	&	$\dot{Q}_{CSPHX}$	&	328	\\
%     CSPHX Effectiveness (-)	&	$\varepsilon_{CSPHX}$	&	0.9368	\\
%     \bottomrule
%     \end{tabular}\\
% \end{specialtable}

% \textbf{Discussion of Results}

% %--------------------------------------------------------------------------------------

% %--------------------------------------------------------------------------------------

% The results from this study are displayed in Table \ref{tab-c-lfr-par}.

% \begin{specialtable}[H]
%     %[htbp]
%     \caption{Calculated system parameters for salt charging C-LFR-PAR cycle configuration with TES cold storage varied and LFR low temperature set to 673.2 K.\label{tab-c-lfr-par}}
%     \begin{tabular}{lccc}
%     \toprule
%     \textbf{Definition} & \textbf{Variable} & \textbf{C-2HTR3T-ON} & \\
%     & & \textit{C} & \textit{C}\\
%     \midrule	
%     Cold TES Temperature ($^{\circ}$C)	&	$T_{CS}$	&	390	&	440	\\
%     LFR Inlet Temperature ($^{\circ}$C)	&	$T_{4}$	&	400	&	400	\\
%     Heat Storage Efficiency (\%)	&	$\eta_{heatstorage}$	&	45.30	&	45.30	\\
%     Alternator Power (MW)	&	$\dot{W}_{A}$	&	0	&	0	\\
%     PC Heat Transfer (MW)	&	$\dot{Q}_{PC}$	&	519.6	&	519.6	\\
%     MC Power (MW)	&	$\dot{W}_{MC}$	&	118.9	&	119	\\
%     RC Power (MW)	&	$\dot{W}_{RC}$	&	77.28	&	77.28	\\
%     Turbine Power (MW)	&	$\dot{W}_{T}$	&	196.2	&	196.2	\\
%     MC Mass Flow Fraction (-)	&	$y_{1}$	&	0.7844	&	0.7844	\\
%     SALT HX Mass Flow Fraction (-)	&	$y_{2}$	&	0.6867	&	0.6867	\\
%     LTR UA Value (MW/$^{\circ}$C)	&	$UA_{LTR}$	&	22.83	&	22.83	\\
%     LTR Capacitance Ratio (-)	&	$CR_{LTR}$	&	0.8473	&	0.8743	\\
%     LTR Heat Transfer Rate (MW)	&	$\dot{Q}_{LTR}$	&	366.4	&	366.4	\\
%     LTR Effectiveness (-)	&	$\varepsilon_{LTR}$	&	0.8742	&	0.8742	\\
%     HTR UA Value (MW/$^{\circ}$C)	&	$UA_{HTR}$	&	42.69	&	42.69	\\
%     HTR Capacitance Ratio (-)	&	$CR_{HTR}$	&	0.8143	&	0.8143	\\
%     HTR Heat Transfer Rate (MW)	&	$\dot{Q}_{HTR}$	&	1160	&	1160	\\
%     HTR Effectiveness (-)	&	$\varepsilon_{HTR}$	&	0.9627	&	0.9627	\\
%     CSPHX UA Value (MW/$^{\circ}$C)	&	$UA_{CSPHX}$	&	8.037	&	14.24	\\
%     CSPHX Capacitance Ratio (-)	&	$CR_{CSPHX}$	&	0.7556	&	0.9339	\\
%     CSPHX Heat Transfer Rate (MW)	&	$\dot{Q}_{CSPHX}$	&	430.4	&	430.4	\\
%     CSPHX Effectiveness (-)	&	$\varepsilon_{CSPHX}$	&	0.8275	&	0.8307	\\
%     CSPHX Approach Temperature ($^{\circ}$C)	&	$\delta_{CSPHX}$	&	35	&	26.27	\\
%     \bottomrule
%     \end{tabular}\\
% \end{specialtable}

% Changing the temperature of the cold CSP TES had little effect on the heat storage efficiency. The CSP salt mass flow rate and approach temperature of the SALT HX would adjust according to the temperature difference in the TES and keep the efficiency constant.


% %--------------------------------------------------------------------------------------

% Table \ref{tab-c-lfr-circ} to show cold thermal energy storage's affect on heat storage efficiency.

% \begin{specialtable}[H]
%     %[htbp]
%     \caption{Calculated system parameters for charging C-LFR-CIRC subcycle configuration with constrained lead-cooled fast reactor low-end temperature.\label{tab-c-lfr-circ}}
%     \begin{tabular}{lcccc}
%     \toprule
%     \textbf{Definition} & \textbf{Variable} & \textbf{C-LFR-CIRC} &\\
%     \midrule	
%     Cold TES Temperature ($^{\circ}$C)	&	$T_{CS}$	&	390	&	410	&	440	\\
%     LFR Inlet Temperature ($^{\circ}$C)	&	$T_{1C}$	&	400	&	400	&	400	\\
%     Heat Storage Efficiency (\%)	&	$\eta_{heatstorage}$	&	99.92	&	89.66	&	74.29	\\
%     Chiller Heat Transfer (MW)	&	$\dot{Q}_{chill}$	&	0.7245	&	98.25	&	244.2	\\
%     CSPHX UA Value (MW/$^{\circ}$C)	&	$UA_{CSPHX}$	&	48.29	&	43.16	&	35.63	\\
%     CSPHX Capacitance Ratio (-)	&	$CR_{CSPHX}$	&	0.8755	&	0.8599	&	0.8294	\\
%     CSPHX Heat Transfer Rate (MW)	&	$\dot{Q}_{CSPHX}$	&	949.3	&	851.8	&	705.8	\\
%     CSPHX Effectiveness (-)	&	$\varepsilon_{CSPHX}$	&	0.9511	&	0.9459	&	0.9355	\\
%     \bottomrule
%     \end{tabular}\\
% \end{specialtable}




%This section may be divided by subheadings. It should provide a concise and precise description of the experimental results, their interpretation as well as the experimental conclusions that can be drawn.


%%%%%%%%%%%%%%%%%%%%%%%%%%%%%%%%%%%%%%%%%%%%%%%%%%%%%%%%%%%%%%%%%%%%%%%%%%%%%%%%%%%%%%%
% \section{Discussion}

% Authors should discuss the results and how they can be interpreted from the perspective of previous studies and of the working hypotheses. The findings and their implications should be discussed in the broadest context possible. Future research directions may also be highlighted.

%%%%%%%%%%%%%%%%%%%%%%%%%%%%%%%%%%%%%%%%%%%%%%%%%%%%%%%%%%%%%%%%%%%%%%%%%%%%%%%%%%%%%%%
\section{Conclusions}

This section is not mandatory, but can be added to the manuscript if the discussion is unusually long or complex.

\end{paracol}

%%%%%%%%%%%%%%%%%%%%%%%%%%%%%%%%%%%%%%%%%%%%%%%%%%%%%%%%%%%%%%%%%%%%%%%%%%%%%%%%%%%%%%%
%\section{how to use}

%\subsection{Subsection}
%Citing a journal paper \cite{wagner2017optimization} . Now citing a book reference \cite{blair2005sam} or other reference types \cite{hirsch2011standardization}. \cite{nellis_klein_2008}
%\subsubsection{Subsubsection}

%Bulleted lists look like this:
%\begin{itemize}
%\item	First bullet;
%\item	Second bullet;
%\item	Third bullet.
%\end{itemize}

%Numbered lists can be added as follows:
%\begin{enumerate}
%\item	First item; 
%\item	Second item;
%\item	Third item.
%\end{enumerate}

%The text continues here. 

%\subsection{Figures, Tables and Schemes}

%All figures and tables should be cited in the main text as Figure~\ref{fig1}, Table~\ref{tab1}, etc.

%\begin{figure}[H]
%\includegraphics[width=10.5 cm]{Definitions/logo-mdpi}
%\caption{This is a figure. Schemes follow the same formatting. If there are multiple panels, they should be listed as: (\textbf{a}) Description of what is contained in the first panel. (\textbf{b}) Description of what is contained in the second panel. Figures should be placed in the main text near to the first time they are cited. A caption on a single line should be centered.\label{fig1}}
%\end{figure}   

% The MDPI table float is called specialtable
%\begin{specialtable}[htbp] 
%\caption{This is a table caption. Tables should be placed in the main text near to the first time they are~cited.\label{tab1}}
%%% \tablesize{} %% You can specify the fontsize here, e.g., \tablesize{\footnotesize}. If commented out \small will be used.
%\begin{tabular}{ccc}
%\toprule
%\textbf{Title 1}	& \textbf{Title 2}	& \textbf{Title 3}\\
%\midrule
%Entry 1		& Data			& Data\\
%Entry 2		& Data			& Data\\
%\bottomrule
%\end{tabular}
%\end{specialtable}

%\begin{listing}[H]
%\caption{Title of the listing}
%\rule{\columnwidth}{1pt}
%\raggedright Text of the listing. In font size footnotesize, small, or normalsize. Preferred format: left aligned and single spaced. Preferred border format: top border line and bottom border line.
%\rule{\columnwidth}{1pt}
%\end{listing}

%\subsection{Formatting of Mathematical Components}

%This is the example 1 of equation:
%\begin{equation}
%a = 1,
%\end{equation}

%the text following an equation need not be a new paragraph. Please punctuate equations as regular text.
%% If the documentclass option "submit" is chosen, please insert a blank line before and after any math environment (equation and eqnarray environments). This ensures correct linenumbering. The blank line should be removed when the documentclass option is changed to "accept" because the text following an equation should not be a new paragraph.

%This is the example 2 of equation:
%\end{paracol}
%\nointerlineskip
%\begin{eqnarray}
%a &=& b + c + d + e + f + g + h + i + j + k + l\nonumber \\
% &+& m + n + o + p + q + r + s + t + u + v + w + x + y + z %\nonumber
%\end{eqnarray}

% Example of a figure that spans the whole page width (the commands \widefigure and \begin{paracol}{2}, \linenumbers, and\switchcolumn need to be present). The same concept works for tables, too.
%\begin{figure}[H]	
%\widefigure
%\includegraphics[width=15 cm]{Definitions/logo-mdpi}
%\caption{This is a wide figure.\label{fig2}}
%\end{figure} 





%\begin{paracol}{2}
%\linenumbers
%\switchcolumn
%Please punctuate equations as regular text. Theorem-type environments (including propositions, lemmas, corollaries etc.) can be formatted as follows:
%% Example of a theorem:
%\begin{Theorem}
%Example text of a theorem.
%\end{Theorem}

%The text continues here. Proofs must be formatted as follows:

%% Example of a proof:
%\begin{proof}[Proof of Theorem 1]
%Text of the proof. Note that the phrase ``of Theorem 1'' is optional if it is clear which theorem is being referred to.
%\end{proof}
%The text continues here.

%\end{paracol}



%Completed suggestions from Ben:

%-I would include the naming scheme upfront and give a table (C-LFR-ON etc.)

%-Here you need to explain what a conventional CSP cycle looks like (briefly). I.e. CSP charges salt, it can discharge when you want it to. Now with our configuration we introduce a 'new' charging mode where heat comes in from the SALT HX as well as from the sun, via the sCO2 Brayton as you say

%-Explain why there is no charging and discharging at same time - because if there is electricity demand heat from LFR would be dispatched direclty rather than thru the SALT HX

%-SALT HX could perhaps be better named sCO2-to-Salt or C2S for short perhaps! Np if you dont want to change it at this late stage




%Completed suggestions from Mike: